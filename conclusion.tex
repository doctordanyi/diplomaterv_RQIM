%-----------------------------------------------------------------------------------------------
\chapter{Conclusion}\label{sect:conclusion}
%-----------------------------------------------------------------------------------------------

During the course of this work the process of calculating camera pose was thoroughly examined.
Each step from getting the input image to calculating the camera pose was discussed.
Usually there were multiple possible solutions for every step of the problem.
These were thoroughly compared, for most parts both the underlying theory and empirical test results were presented.
Based on those, the optimal candidates for the current application were selected.

In line with the project targets, multiple pose estimation methods were discussed and compared.
The EP$n$P\cite{Lepetit2008}, an iterative approach\cite{iterative} and the robust pose estimation\cite{robust} were considered as candidates to be used in this project.
Short summaries of their operating principles were presented.
Their performance then was compared with respect to multiple properties: accuracy, robustness, and computational efficiency.
As a result, two algorithms were selected depending on the available computational power available on the target platform.
EP$n$P is recommended for use on mobile devices or embedded platforms, where efficient, non-iterative solution is preferred.
However, if robust and accurate results are necessary (and there are enough resources), the robust pose estimation algorithm is the better choice.

Another focus of the project was developing a marker with advantageous properties for pose estimation.
To achieve this goal the RQIM was proposed.
It is a randomly generated marker put together from multiple quads.
It has been shown to have desirable properties for pose estimation: scale invariance and redundancy.
A formal representation of the quads (both mathematical and computational) have been defined.
A simple algorithm was proposed for generating random markers.
The notion of creating discrete parameter space for quads has also been examined.
It would provide additional robustness and the ability to encode meta-information in the markers, however these possibilities have not been tested.

The main part of this work is dedicated to the development and testing of a marker detecting solution.
It was shown that marker detection is (from an image processing point of view) equivalent to detecting individual quads on the source image.
Two different approaches were made to quad detection: line detection-, and corner detection based solution.
Multiple line detection algorithms were considered for use.
To make a more informed decision on which one to use, their theoretical foundations have been summarised.
The theory of corner detection was also covered.
Four different quad detectors were developed and tested: each based on a different underlying algorithm.
Their implementation details were discussed and their python source code is published in the appendix.

The quad detectors were not only compared based on the theoretical capabilities of their underlying algorithm, they were also extensively tested.
A testing methodology have been developed for comparing the algorithms.
The test were run on randomly generated quads with fixed sizes.
For each size, a 1000 instance was generated to guarantee that the results are statistically significant.
To quantify the detection error, multiple error functions were defined.
The detectors were compared by the ones that describe the overall error in detection.
Error functions were defined for each quad parameter; with their help, the distribution of inaccuracy between the parameters were examined.

Based on the data obtained by the tests, the LSD\cite{LSDDet} line detector based implementation was selected.
It proved to be the most accurate and resistant to noise.
It is also an efficient algorithm that runs in linear time.

The final part of this work was about organising the above components into a complete pose estimation solution.
Images used as input for pose estimation have to be preprocessed and filtered for noise.
These steps of the processing pipeline have been described in chapter 4.
The detected quad structures of a marker are used as input for the pose estimation algorithms selected in chapter 1.
A RANSAC-like approach was proposed for finding the correspondences between the detected and the original quads.
When the correspondence is found, the pose is further refined using all available point pairs.

As a side note, a method for identifying the detected marker was also proposed.
It is based on the same principle as the detection itself, only extended to check quad pairing with multiple, already known possible markers.
This is however only a first draft for marker identification, it has limitations that make practical application questionable.

To sum up, in this paper a specialised marker for pose estimation was described.
A solution for camera pose estimation based on those markers was developed.
The solution uses the \textit{Robust Pose Estimation for Planar Targets} algorithm for calculating the pose from point correspondences when robustness is needed and there are ample resources.
When the solution is running on a platform with more limitations, the \textit{EPnP} algorithm is recommended for use.
The quad detection is based on line detection, using the \textit{LSD: A Line Segment Detector} algorithm.

There are still many possible ways to improve the each part of this project, these could be addressed by further future research.
Firstly, the markers themselves could be refined.
The generation process could be more efficient, or more flexible.
Experiments could be made with selecting the optimal ranges for the quad parameters to better cover the marker area.
Although it was only lightly touched in this work, the field of discrete RQIM holds quite a lot of possibilities.
More robust and efficient detection algorithm could be build around them.
It would be also possible to code extra information to the discrete quad parameters.
Another possibility for expanding the parameter space (or to encode meta information to help the detection process) is to add some kind of colour-coding to the marker.
This could be used for example to simplify the finding of corresponding quads in the detected and original marker.

The quad detection algorithm could also be improved.
More detection principles could be examined.
Just to name one, quad objects could directly be fit to the pixels of the preprocessed region.
However, the detectors described in this work could also be improved in various ways.
As a starting points, none of them were fully prepared for every possible edge-case.
That is, there are situations (although not many), where a quad could probably be detected, but the implementation does not handle it.
As it was stated in chapter 3, the detector based on the probabilistic Hough transform should be refined, as it comes close to the results achieved by LSD.

The preprocessing steps are now limited to dealing with Gaussian White Noise.
However, on real images more error classes are present.
Further noise filtering could be implemented.

Last, but not least, the functionality of the solution could be extended.
For example, in this work it was supposed that the camera used is calibrated.
In the future, camera calibration could also be made to be part of the solution.