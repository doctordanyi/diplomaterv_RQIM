%-----------------------------------------------------------------------------------------------
\chapter{Conclusion}\label{sect:conclusion}
%-----------------------------------------------------------------------------------------------

During the course of this work the process of calculating camera pose was thoroughly examined.
Each step from getting the input image to calculating the camera pose was discussed.
For most parts both the underlying theory and empirical test results were presented.

In line with the project targets, multiple pose estimation methods were discussed and compared.
The EP$n$P\cite{Lepetit2008}, an iterative approach\cite{iterative} and the robust pose estimation\cite{robust} were considered as candidates to be used in this project.
Short summaries of their operating principles were presented.
Their performance then was compared with respect to multiple properties: accuracy, robustness, and computational efficiency.
As a result, two algorithms were selected depending on the available computational power available on the target platform.
EP$n$P is recommended for use on mobile devices or embedded platforms, where efficient, non-iterative solution is preferred.
However, if robust and accurate results are necessary (and there are enough resources), the robust pose estimation algorithm is the better choice.

Another focus of the project was developing a marker with advantageous properties for pose estimation.
To achieve this goal the RQIM was proposed.
It is a randomly generated marker put together from multiple quads.
It has been shown to have desirable properties for pose estimation: scale invariance and redundancy.
A formal representation of the quads (both mathematical and computational) have been defined.
A simple algorithm was proposed for generating random markers.
The notion of creating discrete parameter space for quads has also been examined.
It would provide additional robustness and the ability to encode meta-information in the markers, however these possibilities have not been tested.

The main part of this work is dedicated to the development and testing of a marker detecting solution.
It was shown that marker detection is (from an image processing point of view) equivalent to detecting individual quads on the source image.
Two different approaches were made to quad detection: line detection-, and corner detection based solution.
Multiple line detection algorithms were considered for use.
To make a more informed decision on which one to use, their theoretical foundations have been summarised.
The theory of corner detection was also covered.
Four different quad detectors were developed and tested: each based on a different underlying algorithm.
Their implementation details were discussed and their python source code is published in the appendix.

The quad detectors were not only compared based on the theoretical capabilities of their underlying algorithm, they were also extensively tested.
A testing methodology have been developed for comparing the algorithms.
The test were run on randomly generated quads with fixed sizes.
For each size, a 1000 instance was generated to guarantee that the results are statistically significant.
To quantify the detection error, multiple error functions were defined.
The detectors were compared by the ones that describe the overall error in detection.
Error functions were defined for each quad parameter; with their help, the distribution of inaccuracy between the parameters were examined.

Based on the data obtained by the tests, the LSD\cite{LSDDet} line detector based implementation was selected.
It proved to be the most accurate and resistant to noise.
It is also an efficient algorithm that runs in linear time.

The final part of this work was about organising the above components into a complete pose estimation solution.
Images used as input for pose estimation have to be preprocessed and filtered for noise.
These steps of the processing pipeline have been described in chapter 4.
The detected quad structures of a marker are used as input for the pose estimation algorithms selected in chapter 1.
A RANSAC-like approach was proposed for finding the correspondences between the detected and the original quads.