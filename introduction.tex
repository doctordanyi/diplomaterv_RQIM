%-----------------------------------------------------------------------------------------------
\chapter*{Introduction}\addcontentsline{toc}{chapter}{Introduction}
%-----------------------------------------------------------------------------------------------

Computer vision, and image processing in general, is a computationally intensive area.
In the past the use of these algorithms was severely limited by the lack of processing power.
Image processing solutions were mostly used for scientific purposes, and the algorithms ran offline: real-time applications were not possible.
Satellite photos were analysed, medical imaging solutions were developed at the time.
Optical character recognition was also a popular topic for image processing research.
A famous scientific example from that time gave the basis for the Hough transformation, which will also be discussed in this work.
The transformation was developed to automatically analyse bubble chamber photographs.

With the developing technology, specifically semiconductor manufacturing, more and more possible uses for image processing began to appear.
Around the 1970s cheaper computers and dedicated hardware solutions started spreading.
This made it possible to create real time image processing applications for some use-cases.
One such use-case was television standards conversion.

As general purpose computers became faster and cheaper, they replaced the specialised circuits in almost all areas of application.
Nowadays image processing solution to common problems (localisation, mapping, measurement, etc...) is chosen as a solution because it became the cheapest and most versatile alternative.
Furthermore, 3D computer vision applications became not only possible, but widespread.
3D scanners, range finders, virtual- and augmented reality solutions have spread from laboratories and research institutions to consumer electronics.
Processing power is no longer a bottleneck for most computer vision applications.

