%-----------------------------------------------------------------------------------------------
\chapter*{Introduction}\addcontentsline{toc}{chapter}{Introduction}
%-----------------------------------------------------------------------------------------------

Computer vision, and image processing in general, is a computationally intensive area.
In the past the use of these algorithms was severely limited by the lack of processing power.
Image processing solutions were mostly used for scientific purposes, and the algorithms ran off-line: real-time applications were not possible.
Satellite photos were analysed, medical imaging solutions were developed at the time.
Optical character recognition was also a popular topic for image processing research.
A famous scientific example from that time gave the basis for the Hough transformation, which will also be discussed in this work.
The transformation was developed to automatically analyse bubble chamber photographs.

With the developing technology, specifically semiconductor manufacturing, more and more possible uses for image processing began to appear.
Around the 1970s cheaper computers and dedicated hardware solutions started spreading.
This made it possible to create real time image processing applications for some use-cases.
One such use-case was television standards conversion.

As general purpose computers became faster and cheaper, they replaced the specialized circuits in almost all areas of application.
Nowadays image processing solution to common problems (localization, mapping, measurement, etc...) is chosen as a solution because it became the cheapest and most versatile alternative.
Furthermore, 3D computer vision applications became not only possible, but widespread.
3D scanners, range finders, virtual- and augmented reality solutions have spread from laboratories and research institutions to consumer electronics.
Processing power is no longer a bottleneck for most computer vision applications.

In this paper a common 3D computer vision problem will be discussed: camera pose estimation.
The result of pose estimation is the location and orientation of the camera in a previously defined world coordinate system.
This information is then used in many areas of application.
The nowadays popular vision-based navigation and mapping (SLAM) solutions are heavily based on knowing the spacial coordinates of the observer.
Another application where the pose estimation problem has to be solved is augmented reality (AR).
In AR an accurate and stable pose estimation solution is a must, otherwise the projected virtual objects would not fit in the observed scene.

From a technical aspect, the problem can be solved in various ways.
As a solution, markers with known structures can be placed in the scene, and the camera pose can be calculated based on the observed properties of those markers.
The markers can be planar or 3-dimensional.
Another class of pose estimation systems use the already present structure of the observed scene.
Those work by tracking recognisable features through time or multiple cameras.
This project focuses on using fiducial, planar markers for solving the pose estimation problem.

The goal of this project is to develop a marker based pose estimation solution.
To achieve this goal, already existing algorithm for estimating the camera pose.
Only algorithms based on corresponding point pairs were examined.
Defining a suitable marker is also within the scope of the project.
A method for using those markers for pose estimation is also developed.
The accuracy and robustness of the developed solution is also a target.

This paper covers the above mentioned goals with the following structure.
The first chapter is dedicated to the pose estimation problem.
A formal definition if the problem will be presented.
Later on, existing solutions to the pose estimation problem will be summarised.
Finally, the algorithms will be compared, and the most suitable for the project will be selected.

The second chapter covers the marker used in this project.
The properties required from a suitable marker will be presented.
Then a marker meeting those criteria will be proposed, along with is mathematical formulation and implementation details.
An algorithm for generating randomised markers will also be presented. 

In chapter 3 various detection techniques will be discussed for the proposed markers.
First, the theoretical foundations of the detection algorithms will be presented.
This will cover several line detection methods and an algorithm for corner recognition.
After the theory has been discussed, marker detection solutions will be built around them.
These alternative solutions will then be benchmarked and compared.
In the last part of the chapter the most suitable detector will be selected.

Chapter four will be about building a pose estimating solution around the marker and the recognition algorithm so far defined.
The preprocessing and noise filtering techniques used for preparing the input image for processing will be explained.
The details of using the detected marker points for pose estimation will also be covered in this chapter.
Lastly, a solution for identifying the marker detected by the process will be proposed.

In the last chapter, a short summary of the findings in this work will be presented.
