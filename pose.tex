%-----------------------------------------------------------------------------------------------
\chapter{Pose estimation}\label{sect:pose}
%-----------------------------------------------------------------------------------------------

The goal of this project is to calculate camera pose using a fiducial marker.
To achieve this, a wide range of problems have to be solved including but not limited to marker design, detection, pose calculation.
This chapter will focus on 3D pose estimation problem.
First, it's mathematical formulation will be presented.
Later some solutions to the problem will be summarised and compared.
Based on this comparison, an algorithm will be selected for use in this project.

Taking photos with a camera can be thought of as mapping 3D points from scene being observed to the image plane of the camera.
This mapping depends on a number of parameters, some of witch is tied to the camera, others depend on the viewpoint and angle.
Based on these parameters, a \textit{perspective projection model} can be constructed.
Mathematically, the model can described by a $P$ projection matrix.
It is a $3*4$ matrix, which gives the image point\footnote{Determined up to a scale factor} for any given world point in homogeneous coordinates.
\begin{equation}
	P = K [R|T]
	\label{eq:projMatrix}
\end{equation}
The $P$ matrix can be constructed as shown in \eqref{projMatrix}.

$K$ is a $3*3$ matrix describing the \textit{intrinsic camera parameters} or \textit{camera model}.
These parameters depend only the internals of the camera, thus are independent from the orientation or position.
Once these are measured for a camera, they can be reused later.
Some models use more variables to describe the internal workings of the camera, others are more simple.
The model used in this work is summarized by the camera matrix described in \eqref{intrinsicParams}.
\begin{equation}
	K =
	\begin{bmatrix}
		f_x & \gamma & u_0 \\
		0   & f_y    & v_0 \\
		0   & 0      & 1
	\end{bmatrix}
	\label{eq:intrinsicParams}
\end{equation}
The notation of the model is the following: $f_x$ and $f_y$ denote the focal lengths of the camera on the $x$ and $y$ axis, while $(u_0,v_0)$ is the principal point.
The principal point is the image point where the optical axis intersects with the image plane.
The parameter $\gamma$ is the skew of the camera, which will be neglected in this work, $\gamma = 0$ will be used.

The other part of the projection matrix, $[R | T]$, describe the position and orientation of the camera in the world coordinate system.
These are called \textit{extrinsic parameters}, and depend on the current configuration (position) of the camera.
The matrix is constructed as shown in \eqref{extParams}.
\begin{equation}
	[R | T] =
	\begin{bmatrix}
		r_{11} & r_{12} & r_{13} & t_1\\
		r_{21} & r_{22} & r_{23} & t_2\\
		r_{31} & r_{32} & r_{33} & t_3\\
	\end{bmatrix}
	\label{eq:extParams}
\end{equation}
It is made up of two separate parts: $R$ describes the orientation, while $T$ the translation with respect to the origin of the world coordinate system.
These parameters are only valid while the camera remains stationary.

The pose estimation problem can be viewed as part of determining the camera parameters.
%-----------------------------------------------------------------------------------------------
\section{Pose Estimation Algorithms}
%-----------------------------------------------------------------------------------------------

%-----------------------------------------------------------------------------------------------
\subsection{Fast and Globally convergent Pose Estimation}
%-----------------------------------------------------------------------------------------------

%-----------------------------------------------------------------------------------------------
\subsection{Linear Pose Estimation from Points or Lines}
%-----------------------------------------------------------------------------------------------

%-----------------------------------------------------------------------------------------------
\subsection{Robust Pose Estimation from a Planar Target}
%-----------------------------------------------------------------------------------------------

%-----------------------------------------------------------------------------------------------
\section{Comparison}
%-----------------------------------------------------------------------------------------------

