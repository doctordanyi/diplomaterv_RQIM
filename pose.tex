%-----------------------------------------------------------------------------------------------
\chapter{Pose estimation}\label{sect:pose}
%-----------------------------------------------------------------------------------------------

The goal of this project is to calculate camera pose using a fiducial marker.
To achieve this, a wide range of problems have to be solved including but not limited to marker design, detection, pose calculation.
This chapter will focus on 3D pose estimation problem.
First, it's mathematical formulation will be presented.
Later some solutions to the problem will be summarised and compared.
Based on this comparison, an algorithm will be selected for use in this project.

The pose estimation problem can be viewed as part of determining the camera parameters.

\begin{equation}
	P = K [R|T]
\end{equation}

\begin{equation}
	K =
	\begin{bmatrix}
		f_x & \gamma & u_0 \\
		0   & f_y    & v_0 \\
		0   & 0      & 1
	\end{bmatrix}
\end{equation}

\begin{equation}
	[R | T] =
	\begin{bmatrix}
		r_{11} & r_{12} & r_{13} & t_1\\
		r_{21} & r_{22} & r_{23} & t_2\\
		r_{31} & r_{32} & r_{33} & t_3\\
	\end{bmatrix}
\end{equation}

%-----------------------------------------------------------------------------------------------
\section{Pose Estimation Algorithms}
%-----------------------------------------------------------------------------------------------

%-----------------------------------------------------------------------------------------------
\subsection{Fast and Globally convergent Pose Estimation}
%-----------------------------------------------------------------------------------------------

%-----------------------------------------------------------------------------------------------
\subsection{Linear Pose Estimation from Points or Lines}
%-----------------------------------------------------------------------------------------------

%-----------------------------------------------------------------------------------------------
\subsection{Robust Pose Estimation from a Planar Target}
%-----------------------------------------------------------------------------------------------

%-----------------------------------------------------------------------------------------------
\section{Comparison}
%-----------------------------------------------------------------------------------------------

