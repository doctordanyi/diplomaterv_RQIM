%-----------------------------------------------------------------------------------------------
\chapter{Marker recognition}\label{sect:markerRec}
%-----------------------------------------------------------------------------------------------

In this chapter there will be a summary of the image processing algorithms tried and used for the recognition of the fiducial markers.
From a computer vision point of view the task is to detect joint line segments.
This is a relatively easy task in image processing, there are many well tried algorithms for it.

In this chapter will be a short summary of the algorithms used for testing and performance comparison.
The preprocessing steps used for preparing the images for the line fitters (segmentation, thresholding, filtering etc.) will also be discussed.

The general flow of processing is the same for every line fitting solution.
The input is the raw image taken\footnote{In the development phase there were rendered pictures used for better repeatability} by the observer.
The first problem is finding the RQIM on the picture.
When the marker area is located, it is necessary to discard the only partially visible and/or unrecognisable quads.
At this point there is an image or set of images containing potentially good quads.
The process here diverges depending on which line fitting algorithm is used.
They all need differently conditioned input images for optimal performance.
The line fitter routines not necessarily have the same output format\footnote{Some return line segments defined by their endpoint, others use the polar representation of a lin, etc.}, so conversion may be needed.
This is the end of the marker recognition phase.
This step of the process takes the raw input image and initiates quad structures based on the observed picture.

Four separate line fitters are profiled in this experiment.
\begin{itemize}
	\item Hough-transformation
	\item Corner detection
	\item Skeletoning detector
	\item Gradient detector (enhanced Hough transformation)
\end{itemize}
The first one uses the classic Hough-transformation for line detection.
In the OpenCV framework there is another, probabilistic implementation of the transformation.
It will also be tested.

The second detector is based on corner recognition.
There are more variants to try out for this method too.
The corner metrics of a feature can be calculated differently with (Harris metric, eigenvalues, etc.) varying results.
It is also needed for the solution to be scale invariant, which also can be achieved in a number of ways.

The third alternative uses skeletoning for line detection.
It is a RANSAC-like method, which makes it very robust, though quite computationally expensive.
The core concept is to thin the observed marker (band) down to single pixel lines and try to find two point with the most inliers.

The gradient detector is a bit Hough-like in it's nature.
It uses the image gradient the calculate the angle in the Hough-space, and the pixels only vote in their distance parameter.

%-----------------------------------------------------------------------------------------------
\section{Preprocessing}
%-----------------------------------------------------------------------------------------------



%-----------------------------------------------------------------------------------------------
\subsection{Segmentation}
%-----------------------------------------------------------------------------------------------



%-----------------------------------------------------------------------------------------------
\subsection{Filtering}
%-----------------------------------------------------------------------------------------------



%-----------------------------------------------------------------------------------------------
\section{Line fitters}
%-----------------------------------------------------------------------------------------------



%-----------------------------------------------------------------------------------------------
\subsection{Hough transformation}
%-----------------------------------------------------------------------------------------------



%-----------------------------------------------------------------------------------------------
\subsection{Gradient detector}
%-----------------------------------------------------------------------------------------------



%-----------------------------------------------------------------------------------------------
\subsection{Corner detector}
%-----------------------------------------------------------------------------------------------



%-----------------------------------------------------------------------------------------------
\subsection{Skeletoning detector}
%-----------------------------------------------------------------------------------------------


