%----------------------------------------------------------------------------
% Abstract in hungarian
%----------------------------------------------------------------------------
\chapter*{Kivonat}\addcontentsline{toc}{chapter}{Kivonat}

\vfill

%----------------------------------------------------------------------------
% Abstract in english
%----------------------------------------------------------------------------
\chapter*{Abstract}\addcontentsline{toc}{chapter}{Abstract}
Image processing has been an intensively researched subject for decades.
Many algorithms that are used today have been developed in the 1960s.
At that time, it was a costly tool mainly used by scientists for satellite imagery, medical imaging, optical character recognition, etc...
The advancement of cheap and powerful general purpose computers opened up new possibilities for research and application.
Real time image processing became possible.
An interesting and even more computationally expensive sub-field of computer vision is 3D reconstruction.
With today's (consumer) technology it is possible to map the 3D world based on image processing solutions.
Navigational, Augmented and Virtual Reality applications are spreading.

This paper will examine the use of fiducial markers for camera pose estimation using a single camera.
The goal of pose estimation is to determine the position and orientation of the camera with respect to a known marker.
This is a complex task, which involves multiple image processing steps, as well as solving optimization problems.
This work will provide an overview of the steps necessary for estimating the camera pose based on picture of a fiducial marker.

A section of this work will be dedicated to the pose estimation problem.
There will be a short summary of the problem of reconstructing the view point based on point pairs in the world coordinate system and image points.
Then some algorithms will be summarised that solved that problem.
This section will be closed by comparing the benefits and drawbacks of these algorithms and choosing the one that best suits the need of this project.

The choice of the marker also influences accuracy and robustness of the pose estimation solution.
There are already some marker types available for use (ARTag, glyph, etc...).
This paper also proposes a new marker type which tries to offer better performance than the aforementioned solutions.
The applicability of the new marker will be examined in various conditions.

The last major part of this work is about the different possible methods for extracting the markers from the images.
In that section there will be short theoretical summaries of the detection methods.
After the theory is covered, implementations of the aforementioned methods will be recommended.
The performance of the detection algorithms will also be benchmarked on optimal and noisy images.
Based on the tests results an optimal method will be selected.
\vfill
